\documentclass[10pt]{article}

\usepackage[margin=1in]{geometry}
\usepackage{amsmath}
\usepackage{amssymb}
\usepackage{amsthm}
\usepackage{mathtools}
\usepackage[shortlabels]{enumitem}
\usepackage[normalem]{ulem}

\usepackage{hyperref}
\hypersetup{
  colorlinks   = true, %Colours links instead of ugly boxes
  urlcolor     = black, %Colour for external hyperlinks
  linkcolor    = blue, %Colour of internal links
  citecolor    = blue  %Colour of citations
}

\usepackage{listings}
\lstset{language=Python} %,numbers=left}

%%%%%%%%%%%%%%%%%%%%%%%%%%%%%%%%%%%%%%%%%%%%%%%%%%%%%%%%%%%%%%%%%%%%%%%%%%%%%%%%

\theoremstyle{definition}
\newtheorem{problem}{Problem}
\newcommand{\E}{\mathbb E}
\newcommand{\R}{\mathbb R}
\DeclareMathOperator{\Var}{Var}
\DeclareMathOperator*{\argmin}{arg\,min}
\DeclareMathOperator*{\argmax}{arg\,max}

\newcommand{\trans}[1]{{#1}^{T}}
\newcommand{\loss}{\ell}
\newcommand{\w}{\mathbf w}
\newcommand{\mle}[1]{\hat{#1}_{\textit{mle}}}
\newcommand{\map}[1]{\hat{#1}_{\textit{map}}}
\newcommand{\normal}{\mathcal{N}}
\newcommand{\x}{\mathbf x}
\newcommand{\y}{\mathbf y}
\newcommand{\ltwo}[1]{\lVert {#1} \rVert}

%%%%%%%%%%%%%%%%%%%%%%%%%%%%%%%%%%%%%%%%%%%%%%%%%%%%%%%%%%%%%%%%%%%%%%%%%%%%%%%%

\begin{document}

\begin{center}
    {
\Large
Week 3 Quiz
}

    \vspace{0.1in}
    CSCI040, \sout{Computing for the Web} Introduction to Hacking

    \vspace{0.1in}
\end{center}

\vspace{0.15in}
%\noindent
%\textbf{Total Score:} ~~~~~~~~~~~~~~~/10
%
%\vspace{0.5in}
%\noindent
%\textbf{Name:} (5pt)
%
%\noindent
%\rule{\textwidth}{0.1pt}
%\vspace{0.25in}

\noindent\textbf{Instructions.}
\begin{enumerate}
    \item
        Complete each problem without using the computer.  
        You may use your Python cheat sheet or any other class notes,
        but you may not type the expressions into Python.
        There is no time limit, but I designed the quiz to be completed in about 10 minutes.
        \emph{You are encouraged to work with other students.}
\item
After you have completed each problem, you will grade your own quiz.
To do this, type the code into an interactive Python session,
and compare the result the Python gives you with the result that you wrote down.
If they match, you get 2 points on the problem,
if they don't match, you get 1 point on the problem.
\item
Scan your quiz and upload it to sakai.
In the text submission, write the score that you achieved.
\end{enumerate}
\vspace{0.15in}

\begin{problem}
    What does the following expression evaluate to?
\end{problem}
\begin{lstlisting}
    2//3*4+5
\end{lstlisting}
\vspace{1.5in}

\begin{problem}
    What does the following expression evaluate to?
\end{problem}
\begin{lstlisting}
    2**3+4%5
\end{lstlisting}
\vspace{1.5in}

\newpage
\begin{problem}
    What is the output of the following code:
\end{problem}
\begin{lstlisting}
    for i in range(10,15,1):
        print('i=',i)
\end{lstlisting}
\vspace{2in}

\begin{problem}
    What is the output of the following code:
\end{problem}
\begin{lstlisting}
    total=0
    for i in range(5):
        total-=1
    print('total=',total)
\end{lstlisting}
\vspace{2in}

\begin{problem}
    What is the output of the following code:
\end{problem}
\begin{lstlisting}
result=1
for i in range(5):
    if i<3:
        result*=1
    else:
        result*=(-1)
print('result=',result)
\end{lstlisting}

\end{document}


