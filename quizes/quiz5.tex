\documentclass[10pt]{article}

\usepackage[margin=1in]{geometry}
\usepackage{amsmath}
\usepackage{amssymb}
\usepackage{amsthm}
\usepackage{mathtools}
\usepackage[shortlabels]{enumitem}

\usepackage{hyperref}
\hypersetup{
  colorlinks   = true, %Colours links instead of ugly boxes
  urlcolor     = black, %Colour for external hyperlinks
  linkcolor    = blue, %Colour of internal links
  citecolor    = blue  %Colour of citations
}

\usepackage{listings}
\lstset{language=Python} %,numbers=left}

%%%%%%%%%%%%%%%%%%%%%%%%%%%%%%%%%%%%%%%%%%%%%%%%%%%%%%%%%%%%%%%%%%%%%%%%%%%%%%%%

\theoremstyle{definition}
\newtheorem{problem}{Problem}
\newcommand{\E}{\mathbb E}
\newcommand{\R}{\mathbb R}
\DeclareMathOperator{\Var}{Var}
\DeclareMathOperator*{\argmin}{arg\,min}
\DeclareMathOperator*{\argmax}{arg\,max}

\newcommand{\trans}[1]{{#1}^{T}}
\newcommand{\loss}{\ell}
\newcommand{\w}{\mathbf w}
\newcommand{\mle}[1]{\hat{#1}_{\textit{mle}}}
\newcommand{\map}[1]{\hat{#1}_{\textit{map}}}
\newcommand{\normal}{\mathcal{N}}
\newcommand{\x}{\mathbf x}
\newcommand{\y}{\mathbf y}
\newcommand{\ltwo}[1]{\lVert {#1} \rVert}

%%%%%%%%%%%%%%%%%%%%%%%%%%%%%%%%%%%%%%%%%%%%%%%%%%%%%%%%%%%%%%%%%%%%%%%%%%%%%%%%

\begin{document}

\begin{center}
    {
\Large
In-class Quiz 5
}

    \vspace{0.1in}
CSCI040, Computing for the Web

    \vspace{0.1in}
\end{center}

\vspace{0.25in}
\noindent
\textbf{Total Score:} ~~~~~~~~~~~~~~~/10

\vspace{0.5in}
\noindent
\textbf{Name:} (2pt)

\noindent
\rule{\textwidth}{0.1pt}
\vspace{0.15in}

\noindent 
\textbf{Note:} The attached handout implements several python functions that are used in the following problems.

\begin{problem}
    (2pt)
    The following code (circle one)
    
    \vspace{0.25in}
    \hspace{0.5in}terminates successfully
    \hspace{1in}runs forever
    \hspace{1in}generates an error
    \vspace{0.25in}

    \noindent
    If the code terminates successfully, what is the output of the code?
    If the code runs forever or generates an error, explain why.
\end{problem}
\begin{lstlisting}
    x = foo(4)
    y = bar(2)
    z = baz(0)
    print('z=',z)
\end{lstlisting}
\vspace{1.5in}

\begin{problem}
    (2pt)
    The following code (circle one)
    
    \vspace{0.25in}
    \hspace{0.5in}terminates successfully
    \hspace{1in}runs forever
    \hspace{1in}generates an error
    \vspace{0.25in}

    \noindent
    If the code terminates successfully, what is the output of the code?
    If the code runs forever or generates an error, explain why.
\end{problem}
\begin{lstlisting}
    x = bar(j=3,i=2)
    y = foo(i=x)
    z = baz()
    print('z=',z)
\end{lstlisting}
\vspace{2in}
\newpage
\begin{problem}
    (2pt)
    The following code (circle one)
    
    \vspace{0.25in}
    \hspace{0.5in}terminates successfully
    \hspace{1in}runs forever
    \hspace{1in}generates an error
    \vspace{0.25in}

    \noindent
    If the code terminates successfully, what is the output of the code?
    If the code runs forever or generates an error, explain why.
\end{problem}
\begin{lstlisting}
    y = foo(bar(1,2))
\end{lstlisting}
\vspace{2.5in}

\begin{problem}
    (2pt)
    The following code (circle one)
    
    \vspace{0.25in}
    \hspace{0.5in}terminates successfully
    \hspace{1in}runs forever
    \hspace{1in}generates an error
    \vspace{0.25in}

    \noindent
    If the code terminates successfully, what is the output of the code?
    If the code runs forever or generates an error, explain why.
\end{problem}
\begin{lstlisting}
    z = baz(baz(baz()))
\end{lstlisting}
\vspace{1.5in}

\newpage

\begin{center}
    {
\Large
In-class Quiz 5 Supplement
}

    \vspace{0.1in}
CSCI040, Computing for the Web

    \vspace{0.1in}
\end{center}

This supplement defines several sets of functions.
Different versions of the quiz are defined by using different sets of functions below.

\noindent
\rule{\textwidth}{0.1pt}

\begin{lstlisting}
    def foo(i):
        if i<10:
            return -i
        else: 
            return i

    def bar(i,j):
        print(i+j)

    def baz(x = 3):
        return foo(x) + bar(x-1,x+1)
\end{lstlisting}

\noindent
\rule{\textwidth}{0.1pt}

\begin{lstlisting}
    def foo(i):
        sum=0
        for j in range(i):
            sum-=j
            print(sum)
        return 'sum'

    def bar(a,b):
        return 'a+b'

    def baz(x = -1):
        return baz(x)
\end{lstlisting}

\noindent
\rule{\textwidth}{0.1pt}

\begin{lstlisting}
    def foo(i):
        for i in range(3):
            print('foo')
        return 3

    def bar(i,j):
        for i in range(3):
            return i+j

    def baz(x = 3):
        return foo(x) + foo(x-1)
\end{lstlisting}

\noindent
\rule{\textwidth}{0.1pt}

\end{document}


