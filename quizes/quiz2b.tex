\documentclass[10pt]{article}

\usepackage[margin=1in]{geometry}
\usepackage{amsmath}
\usepackage{amssymb}
\usepackage{amsthm}
\usepackage{mathtools}
\usepackage[shortlabels]{enumitem}

\usepackage{hyperref}
\hypersetup{
  colorlinks   = true, %Colours links instead of ugly boxes
  urlcolor     = black, %Colour for external hyperlinks
  linkcolor    = blue, %Colour of internal links
  citecolor    = blue  %Colour of citations
}

\usepackage{listings}
\lstset{language=Python} %,numbers=left}

%%%%%%%%%%%%%%%%%%%%%%%%%%%%%%%%%%%%%%%%%%%%%%%%%%%%%%%%%%%%%%%%%%%%%%%%%%%%%%%%

\theoremstyle{definition}
\newtheorem{problem}{Problem}
\newcommand{\E}{\mathbb E}
\newcommand{\R}{\mathbb R}
\DeclareMathOperator{\Var}{Var}
\DeclareMathOperator*{\argmin}{arg\,min}
\DeclareMathOperator*{\argmax}{arg\,max}

\newcommand{\trans}[1]{{#1}^{T}}
\newcommand{\loss}{\ell}
\newcommand{\w}{\mathbf w}
\newcommand{\mle}[1]{\hat{#1}_{\textit{mle}}}
\newcommand{\map}[1]{\hat{#1}_{\textit{map}}}
\newcommand{\normal}{\mathcal{N}}
\newcommand{\x}{\mathbf x}
\newcommand{\y}{\mathbf y}
\newcommand{\ltwo}[1]{\lVert {#1} \rVert}

%%%%%%%%%%%%%%%%%%%%%%%%%%%%%%%%%%%%%%%%%%%%%%%%%%%%%%%%%%%%%%%%%%%%%%%%%%%%%%%%

\begin{document}

\begin{center}
    {
\Large
In-class Quiz 2b (practice, do not turn in)
}

    \vspace{0.1in}
CSCI040, Computing for the Web

    \vspace{0.1in}
\end{center}

\vspace{0.25in}
\noindent
\textbf{Total Score:} ~~~~~~~~~~~~~~~/10

\vspace{0.5in}
\noindent
\textbf{Name:} (4pt)

\noindent
\rule{\textwidth}{0.1pt}
\vspace{0.25in}

\begin{problem}
    (2pt)
    What is the output of the following code:
\end{problem}
\begin{lstlisting}
    total=0
    for i in range(3):
        for j in range(i):
            total+=j
    print('total=',total)
\end{lstlisting}
\vspace{2in}

\newpage
\begin{problem}
    (2pt)
    Convert the following for loop into an equivalent while loop.
    (This involves writing some code, not just shoring the output.)
\end{problem}
\begin{lstlisting}
    for i in range(5,10,2):
        print('i=',i)
\end{lstlisting}
\vspace{3.5in}

\begin{problem}
    (2pt)
    Convert the following while loop into a for loop.
    (This involves writing some code, not just shoring the output.)
\end{problem}
\begin{lstlisting}
    i=0
    while True:
        print('i=',i)
        if i>10:
            break
        i+=3
\end{lstlisting}
\vspace{1.5in}

\end{document}



